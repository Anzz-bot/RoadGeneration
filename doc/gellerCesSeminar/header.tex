%!TEX root = main.tex
\usepackage[T1]{fontenc}
\usepackage[utf8]{inputenc}
\usepackage[ngerman]{babel}
%\usepackage[scaled=1]{uarial} % set default font to arial
\renewcommand{\familydefault}{\sfdefault}
\usepackage{bm} 
\usepackage{multirow}
\usepackage{float}

% enable hyperlinks in pdf documennt
\usepackage[hidelinks]{hyperref}
%\usepackage[colorinlistoftodos]{todonotes}\reversemarginpar

% Import the stuff for the glossary
\usepackage[numberedsection,nonumberlist,acronym,toc,nopostdot,nomain]{glossaries}
\newglossary[slg]{symbolslist}{syi}{syg}{Formelzeichen und Indizes} % create add. symbolslist
\makeglossaries
\setacronymstyle{long-short}
\loadglsentries[acronyms]{acronym_entries.tex}
\loadglsentries[symbolslist]{symbol_entries.tex}

%% the following commands are needed for some matlab2tikz features
%\usetikzlibrary{plotmarks}
%\usetikzlibrary{arrows.meta}

\usepackage{graphicx}
\usepackage{tkz-euclide}
\usetkzobj{all}
\usepackage{pgfplots}
\usepackage{tikzscale}
\pgfplotsset{
    every tick label/.append style = {/pgf/number format/assume math mode=true,font=\sffamily},
}
\pgfplotsset{max space between ticks=50}
\pgfplotsset{compat=newest}


\usepackage{grffile}
\usepackage{amsmath}
\usepackage{listings}
\usepackage{rotating}
\usepackage{cleveref}
%\crefname{equation}{eq.}{eqs.}
%\Crefname{equation}{Eq.}{Eqs.}
\crefformat{equation}{Gl.~#2#1#3}
\Crefformat{equation}{Gl.~#2#1#3}
%\crefformat{subfigure}{Fig.~#2(#1)#3}
\crefname{table}{Abb.}{Tab.}
\crefname{figure}{Abb.}{Abb.}
\Crefname{table}{Tab.}{Tab.}
\crefname{chapter}{Kap.}{Kap.}
\Crefname{chapter}{Kap.}{Kap.}
\crefname{appendix}{Anhang}{}
\crefname{section}{Kap.}{Kap.}
\Crefname{section}{Kap.}{Kap.}
\crefname{subsection}{Kap.}{Kap.}
\Crefname{subsection}{Kap.}{Kap.}
\crefname{subsubsection}{Kap.}{Kap.}
\Crefname{subsubsection}{Kap.}{Kap.}

\usepackage{array}
\usepackage{booktabs}
\usepackage{color}
\usepackage{tensor}
\usepackage{pdfpages}
\usepackage{bm}
\usepackage[section]{placeins}

% set page margins etc.
\usepackage[
	headheight=1.25cm,
	headsep=1.25cm,
	left=3.0cm,
	right=2.0cm,
	top=3.0cm,
	bottom=2.5cm]{geometry}
	
\usepackage[labelformat=simple]{subcaption}
\usepackage[onehalfspacing]{setspace}
\AtBeginEnvironment{tabular}{\onehalfspacing}

\usepackage{amsmath}
\usepackage{amssymb}
\usepackage{siunitx}
\usepackage{tabto}
\usepackage{pbox}

%%%%%%%%%%%%%%%%%%%%%%%%%
% IKA - format	        %
%%%%%%%%%%%%%%%%%%%%%%%%%

%% labelin
%\newcommand{\reffig}[1]{Fig.~\ref{#1}}
%\newcommand{\refsec}[1]{Chapter~\ref{#1}}
%\newcommand{\refapp}[1]{Appendix~\ref{#1}}
%\newcommand{\refeq}[1]{Eq.~\ref{#1}}

% Anpassung der Gleichungsbeschriftung an das IKA-Layout, e.g. "Gl. 1-1"
\makeatletter
	\def\@eqnnum{{\normalfont \normalcolor Gl.\quad \theequation}}
	\def\tagform@#1{\maketag@@@{Gl.\quad\ignorespaces#1\unskip\@@italiccorr}}
\makeatother

% Anpassung der Tabellen- und Bildbeschriftung an das IKA-Layout, e.g. "Abb. 1-1:	" (Teil 2)
\usepackage[titles]{tocloft}
\renewcaptionname{ngerman}{\figurename}{Abb.}
\renewcommand{\cftfigpresnum}{Abb. }
\renewcommand{\cftfigaftersnum}{:}
\settowidth{\cftfignumwidth}{Abb. 10\quad}
\renewcaptionname{ngerman}{\tablename}{Tab.}
\renewcommand{\thefigure}{{\thechapter-\arabic{figure}}}
\renewcommand{\thetable}{{\thechapter-\arabic{figure}}}
\renewcommand{\theequation}{{\thechapter-\arabic{equation}}}

% Durchgehende Nummerierung f�r Abbildungen und Tabellen: IKA fordert Tabllen mit Abb-Caption: Abb. X-Y
\makeatletter
\let\c@table\c@figure
\makeatother


% Anpassung der Beschriftungen
\usepackage[
	style=base,
	font=onehalfspacing,
	format=hang,
	margin=0cm,
	singlelinecheck=false,
	skip=6pt
	]{caption}

% Anpassung der Abs�tze vor und nach einer figure Umgebung
\setlength{\intextsep}{12pt}

\renewcommand{\thesubfigure}{(\roman{subfigure})}


% Kopf- und Fu�zeile
%%%%%%%%%%%%%%%%%%%%%%
%\usepackage[
%	headsepline=0.75pt,
%	plainheadsepline=true,
%	footsepline=false,
%	plainfootsepline
%	]{scrlayer-scrpage} % F�r die Kopf- und Fu�zeile inkl. Trennlinie und Titel gro� geschrieben
%

%
%\ihead*[\thechapter \tabto{1cm}{\leftmark}]{\thechapter \tabto{1cm}{\leftmark}}
%\cohead*[]{}
%\rohead*[\thepage]{\thepage}
%\ifoot*[]{}
%\cofoot*[]{}
%\rofoot*[]{}
\usepackage{fancyhdr}

\pagestyle{fancy}
\renewcommand{\chaptermark}[1]{\markboth{#1}{}}

\fancypagestyle{ika}{%
    \fancyhf{}
    \lhead{\thechapter \tabto{1cm} \leftmark}
    \rhead{\thepage}
    \fancyfoot[]{}
}
\fancypagestyle{plain}{%
    \fancyhf{}
    \lhead{\leftmark}
    \rhead{\thepage}
    \fancyfoot[]{}
}
\fancypagestyle{apx}{%
    \fancyhf{}
    \lhead{Appendix - \leftmark}
    \rhead{\thepage}
    \fancyfoot[]{}
}
\renewcommand{\headrulewidth}{0.75pt}


% XML Listing
\definecolor{gray}{rgb}{0.4,0.4,0.4}
\definecolor{darkblue}{rgb}{0.0,0.0,0.6}
\definecolor{cyan}{rgb}{0.0,0.6,0.6}

\lstset{
    basicstyle=\ttfamily,
    columns=fullflexible,
    frame = single, 
    numbers = left,
    showstringspaces=false,
    commentstyle=\color{gray}\upshape
}

\lstdefinelanguage{XML}
{
    morestring=[b]",
    morestring=[s]{>}{<},
    morecomment=[s]{<?}{?>},
    stringstyle=\color{black},
    identifierstyle=\color{darkblue},
    keywordstyle=\color{cyan},
    morekeywords={}% list your attributes here
}
\pgfkeys{/pgf/number format/.cd,1000 sep={}}
%%%%%%%%%%%%%%%%%%%%%%%%%%%%%%%%%%
% Formatierung der �berschriften
%%%%%%%%%%%%%%%%%%%%%%%%%%%%%%%%%%

% Kapitel�berschrift
\RedeclareSectionCommands[
  afterskip=4pt,
  beforeskip=4pt
]{chapter}

% Unterkapitel
\RedeclareSectionCommands[
  afterskip=4pt,
  beforeskip=4pt
]{section}

\RedeclareSectionCommands[
  afterskip=4pt,
  beforeskip=4pt
]{subsection}

% Unterkapitel
\RedeclareSectionCommands[
  afterskip=4pt,
  beforeskip=4pt
]{subsubsection}

\RedeclareSectionCommands[
  tocindent=0.3cm
]{section}

\RedeclareSectionCommands[
  tocindent=0.6cm
]{subsection}

\RedeclareSectionCommands[
  tocindent=0.9cm
]{subsubsection}

%\RedeclareSectionCommands[
%  tocbeforeskip=5pt
%]{section,subsection,subsubsection}

% Tabstopp f�r �berschriften auf 1cm setzen
\renewcommand*{\chapterformat}{%
	\makebox[1cm][l]{\chapappifchapterprefix{\nobreakspace}\thechapter
	\IfUsePrefixLine{}}}
\renewcommand*{\sectionformat}{%
	\makebox[2cm][l]{\thesection}}
\renewcommand*{\subsectionformat}{%
	\makebox[2cm][l]{\thesubsection}}
\renewcommand*{\subsubsectionformat}{%
	\makebox[2cm][l]{\thesubsubsection}}

% Schriftarten
\setkomafont{chapter}{\normalfont\bfseries}
\setkomafont{section}{\normalfont\bfseries}
\setkomafont{subsection}{\normalfont\bfseries}
\setkomafont{subsubsection}{\normalfont\bfseries}
\setkomafont{pagehead}{\normalfont}
\setkomafont{chapterentry}{\normalfont}


% RWTH Farbdefinitionen
\definecolor{blueRWTH}{cmyk}{1.0,0.5,0,0}
\definecolor{blueLightRWTH}{cmyk}{0.75,0.38,0,0}
\definecolor{blueLighterRWTH}{cmyk}{0.45,0.14,0,0}

\definecolor{greenRWTH}{cmyk}{0.7,0.0,1.0,0}
\definecolor{greenLightRWTH}{cmyk}{0.525,0,0.75,0}
\definecolor{greenLighterRWTH}{cmyk}{0.35,0,0.5,0}

\definecolor{orangeRWTH}{cmyk}{0,0.4,1.0,0}
\definecolor{orangeLightRWTH}{cmyk}{0,0.3,0.75,0}
\definecolor{orangeLighterRWTH}{cmyk}{0,0.2,0.5,0}

\definecolor{redRWTH}{cmyk}{0.15,1,1,0}
\definecolor{redLightRWTH}{cmyk}{0.1125,0.75,0.75,0}
\definecolor{redLighterRWTH}{cmyk}{0.075,0.5,0.5,0}
\definecolor{rwthLight}{cmyk}{0.0,0.0,0.0,0.5}

% no pagebreak
\usepackage{etoolbox}
\makeatletter
\patchcmd{\scr@startchapter}{\if@openright\cleardoublepage\else\clearpage\fi}{}{}{}
\makeatother